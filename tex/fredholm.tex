\chapter{Условия фредгольмовости разностных операторов}\label{sec:fredhlom}
\section{Основные результаты}
В этом параграфе получены необходимые и достаточные условия фредгольмовости разностного оператора вида \eqref{eq:d_def}, т.е. оператора
\begin{gather*}
    \mathcal A \colon l^p  \to l^p, \\
    (\mathcal A x)(k) = x(k + N) + C_1(k) x(k + N - 1) + \dotsc + C_N(k) x(k), \\
    k \in \groupz, \; x \in l^p = l^p(\groupz, Y), \; p \in [1, \infty].
\end{gather*}

Условия получены на основе сопоставления разностному оператору $\mathcal A$ порядка $N$ разностного оператора первого порядка $ \mathbb A\colon l^p(\groupz, Y^N) \to l^p(\groupz, Y^N)$, определенного формулой \eqref{eq:difference_first_order}, где $\mathcal C_0(k)$ --- тождественный оператор в $Y^N$ при любом $k \in \groupz$. Эти условия описываются с использованием понятия экспоненциальной дихотомии дискретного семейства эволюционных оператоов, которое строится по операторной функции $\mathcal C_1\colon \groupz \to \End{Y^N}$.

Рассмотрим разностный оператор первого порядка $\mathbb D$ из $\End l^p(\groupz, X)$ определенный формулой
\[ (\mathbb D x)(n) = x(n) - U(n)x(n - 1), \quad n \in \groupz, x \in l^p(\groupz, X), \]
где $U \in l^\infty(\groupz, X)$, а $X$ --- комплексное банахово пространство.

По функции $U$ построим дискретное семейство эволюционных операторов
\[\mathcal U \colon \Delta = \menge{(n, m) \in \groupz \times \groupz : m \leq n} \to \End X, \]
определенное равенствами
\[
    \mathcal U (n, m) = \begin{cases}
        U(n)U(n-1)\dotsc U(m+1), & m < n, \\
        I,                       & m = n,
    \end{cases}
\]
где $m, n \in \groupz$.

\begin{definition}\label{def:dich}
    Будем говорить, что семейство эволюционных операторов $\mathcal U$ допускает \emph{экспоненциальную дихотомию} на множестве $\mathbb J \subset \groupz$, если существуют ограниченная проекторнозначная функция $P \colon \mathbb J \to \End X$ и постоянные $M_0 \geq 1$, $\gamma > 0$ такие, что выполнены следующие условия
    \begin{enumerate}
        \item $\mathcal U(n, m) P(m) = P(n) \mathcal U(n, m)$, для всех $m \leq n$, $m, n \in \mathbb J$;
        \item $\norm{\mathcal U(n, m) P(m)} \leq M_0 \exp(-\gamma (n - m))$, для всех $m \leq n$, $m, n \in \mathbb J$;
        \item для $m < n$, $m, n \in \mathbb J$, сужение $\mathcal U_{n, m}: X'(m) \to X'(n)$ оператора $\mathcal U(n, m)$ на область значений $X'(m) = \im Q(m)$ дополнительного проектора $Q(m) = I - P(m)$ есть изоморфизм подпространств $X'(m)$ и $X'(n) = \im Q(n)$.
        Тогда полагаем оператор $\mathcal U(m, n)$ равным оператору $\mathcal U_{n, m}^{-1}$ на $X'(n)$ и равным нулевому оператору на $X(n) = \im P(n) \subset X$.
        \item $\norm{\mathcal U(m, n)} \leq M_0 \exp(\gamma (m - n))$ для всех $m \leq n$ из $\mathbb J$.
    \end{enumerate}
\end{definition}
Пару проекторнозначных функций $P, Q\colon \mathbb J \to \End X$, участвующих в определении \ref{def:dich}, назовём \emph{расщепляющей парой} для семейства $\mathcal U$. Если $P= 0$ или $Q = 0$, то будем говорить, что для $\mathcal U$ имеет место \emph{тривиальная экспоненциальная дихотомия} на $\mathbb J$.

\begin{theorem}[(\cite{BasPas01}, \cite{bohr2005})]\label{th:dich}
    Для того чтобы оператор $\mathbb D \in \End l^p(\groupz, X)$, определяемый функцией $U \in l^\infty(\groupz, X)$, был обратим, необходимо и достаточно, чтобы семейство эволюционных операторов $\mathcal U$ допускало экспоненциальную дихотомию на $\groupz$. Если оператор $\mathbb D$ обратим, то обратный к нему определяется формулой
    \[ (\mathbb D^{-1} y)(n) = \sum_{m=-\infty}^\infty G(n, m) y(m), \quad n \in \groupz, y \in l^p(\groupz, X),\]
    где функция Грина $G\colon \groupz^2 \to \End X$ имеет вид
    \[ G(n, m) = \begin{cases}
        \mathcal U(n, m) P(m), & m \leq n, \\
        -\mathcal U(n, m) Q(m),& m > n,
    \end{cases}, \quad m, n \in \groupz. \]
\end{theorem}

Этот результат для случая $p=\infty$ имеется в монографии Д. Хенри \cite{henri} (в статье \cite{BasPas01} была устранена неточность в доказательстве аналога теоремы \ref{th:dich} из этой монографии).

Далее используется
\begin{assumption}\label{assum:knot}
    Существуют числа $a, b \in \groupz$, $a \leq b$, такие, что семейство эволюционных операторов $\mathcal U$ (построенное по функции $U\colon \groupz \to \End X$) допускает экспоненциальную дихотомию на множествах $\groupz_{-, a} = \menge{n \in \groupz : n \leq a}$, $\groupz_{b,+} = \menge{n \in \groupz : n \geq b}$ с расщепляющими парами проекторнозначных функций
    \begin{align*}
        P_{-}, Q_{-} &\colon \groupz_{-,a} \to \End X,\\
        P_{+}, Q_{+} &\colon \groupz_{b,+} \to \End X.
    \end{align*}
\end{assumption}

Определим оператор $\mathcal N_{b, a} \colon \im Q_{-}(a) \to \im Q_{+}(b)$, равенством
\[ \mathcal N_{b, a} x = Q_{+}(b) \mathcal U(b, a) x, \quad x \in \im Q_{-}(a).\]

Этот оператор введён в рассмотрение в статьях \cite{Bas00}, \cite{Bas13} и назван <<узловым>>. Важность его обусловлена тем, что он действует между подпространствами <<фазового>> пространства $X$, а не в $l^p(\groupz, X)$.

Имеет место следующая теорема (\cite{Bas00}, \cite{Bas13}).
\begin{theorem}\label{th:knot_stinv}
    Пусть для семейства эволюционных операторов $\mathcal U \colon \Delta \hm\to \End X$, построенным по функции $U\colon \groupz \to \End X$ выполнены условия предположения \ref{assum:knot}.

    Тогда имеет место равенство
    \[ \Stinv{\mathbb D} = \Stinv{\mathcal N_{b, a}}.\]

    В частности, для фредгольмовости разностного оператора $\mathbb D$, необходимо и достаточно, чтобы узловой оператор $\mathcal N_{b, a} \colon \im Q_{-}(a) \to \im Q_{+}(b)$ являлся фредгольмовым оператором. При условии фредгольмовости узлового оператора имеют иместо равенства:
    \begin{gather*}
        \dim \ker \mathbb D = \dim \ker \mathcal N_{b, a}, \quad
        \codim \im \mathbb D = \codim \im \mathcal N_{b, a}, \\
        \ind \mathbb D = \ind \mathcal N_{b, a}.
    \end{gather*}
\end{theorem}

Рассмотрим разностный оператор (см. формулу \eqref{eq:difference_short})
\begin{gather*}
\mathbb D = \mathbb S^{-1} \mathbb A = \mathbb S^{-1}(\mathbb S + \mathbb A_1) \in \End{l^p(\groupz, Y^N)}\\
(\mathbb D x)(n) = x(n) + \mathcal C_1(n) x(n-1), \quad x \in l^p(\groupz, Y^N).
\end{gather*}

Заметим, что его состояния обратимости совпадают с состояниями обратимости оператора $\mathcal A$.

\begin{definition}
    Семейство эволюционных операторов $\mathcal U_1 \colon \groupz^2 \to \End{Y^N}$, построенное по функции $-\mathcal C_1 \colon \groupz \to \End{Y^N}$, назовем \emph{семейством эволюционных операторов для однородного разностного уравнения}
    \begin{equation}\label{eq:difference_nth_hom}
        x(k + N) + C_1(k) x(k + N - 1) + \dotsc + C_{N}(k) x(k) = 0, \quad k \in \groupz, \; x \in l^p(\groupz, Y).
    \end{equation}
\end{definition}

Из теорем \ref{th:dich} и \ref{th:knot_stinv} получаем следующие утверждения.

\begin{theorem}
    Для того чтобы разностный оператор $\mathcal A\in \End{l^p}$, $p \in [1, \infty]$, определенный формулой \eqref{eq:d_def}, был обратим, необходимо и достаточно, чтобы семейство эволюционных операторов $\mathcal U_1$, построенное для разностного уравнения \eqref{eq:difference_nth_hom}, допускало экспоненциальную дихотомию на $\groupz$.
\end{theorem}

\begin{theorem}\label{th:fredholm_difference}
    Пусть для семейства $\mathcal U = \mathcal U_1\colon \Delta \to \End Y^N$, построенного для разностного уравнения \eqref{eq:difference_nth_hom}, выполнены условия предположения \ref{assum:knot}.

    Тогда имеет место равенство
    \[ \Stinv{\mathcal A} = \Stinv{\mathcal N_{b, a}}.\]

    В частности, для фредгольмовости разностного оператора $\mathcal A\in \End{l^p}$ необходимо и достаточно, чтобы узловой оператор $\mathcal N_{b, a}$ являлся фредгольмовым оператором. При условии фредгольмовости узлового оператора имеют место равенства:
    \begin{gather*}
        \dim \ker \mathcal A = \dim \ker \mathcal N_{b, a}, \quad
        \codim \im \mathcal A = \codim \im \mathcal N_{b, a}, \\
        \ind \mathcal A = \ind \mathcal N_{b, a}.
    \end{gather*}
\end{theorem}

Отметим, что в условиях теоремы \ref{th:fredholm_difference} узловой оператор $\mathcal N_{b, a}$ действует между подпространствами банахова пространства $Y^N$.

\begin{corollary}
    Если оператор $\mathcal A$ фредгольмов в одном из пространств $l^p$, $p \in [1, \infty]$, то он фредгольмов и в остальных, и его индекс не зависит от значения $p$.
\end{corollary}

В условиях следующей теоремы будем использовать следующее
\begin{assumption}\label{assum:limits}
    Существуют пределы
    \[ \lim_{n\to\pm\infty} C_i(n) = C_i^\pm \in \End X, \quad i = \overline{1,N}. \]
\end{assumption}

Под спектром операторного пучка
\[L^\pm(\lambda) = \lambda^N + C_1^\pm \lambda^{N-1} + \dotsc + C_N^\pm, \quad \lambda \in \fieldc, \]
будем понимать множество таких комплексных чисел $\lambda$, что $L^\pm(\lambda)$ --- необратимый в $\End Y$ оператор.

\begin{theorem}
    В условиях предположения \ref{assum:limits} разностный оператор $\mathcal A$ обратим, если спектральные радиусы $r(L^\pm) = \max\menge{\absv{\lambda} : \lambda \in \spectrum{L^\pm}}$ операторных пучков $L^\pm$ меньше единицы.
\end{theorem}

\begin{proof}
    Непосредственно из теоремы \ref{th:main_abstract} получаем, что спектры $\spectrum{\mathcal C_1^\pm}$ операторов $\mathcal C_1^\pm \in \End Y^N$, заданных матрицами
    \[\mathcal C_1^\pm \sim \begin{pmatrix}
    0 & -I & 0  & \cdots & 0 & 0 \\
    0 & 0  & -I & \cdots & 0 & 0 \\
    0 & 0  & 0 & \cdots & 0 & 0 \\
    \vdots & \vdots & \vdots & \ddots & \vdots & \vdots \\
    0 & 0 & 0 & \cdots & 0 & -I \\
    C_N^\pm & C_{N-1}^\pm & C_{N-2}^\pm & \cdots & C_2^\pm & C_1^\pm
   \end{pmatrix},\]
    совпадают со спектрами $\spectrum{L^\pm}$ операторных пучков $L^\pm$, поэтому $r(\mathcal C_1^\pm) \hm= r(L^\pm) < 1$.
    Операторы $\mathcal C_1^\pm$ являются пределами последовательности $\mathcal C_1(n)$ в равномерной операторной топологии.
    Тогда из \cite[теорема 3]{Bas00} следует, что оператор $\mathbb D$ обратим, следовательно обратим и оператор $\mathcal A$.
\end{proof}

Пусть теперь оператор $\mathcal A \in \End{l^p}$ имеет вид:
\begin{gather*}
    (\mathcal A x)(k) = C_0 x(k + N) + C_1 x(k + N - 1) + \dotsc + C_N x(k), \\
    k \in \groupz, \; x \in l^p = l^p(\groupz, Y), \; p \in [1, \infty],
\end{gather*}
то есть $C_i(k) \equiv C_i \in Y^N$, $i = \overline{0,N}$ --- постоянные функции. В этом случае разностный оператор $\mathbb A$ задан выражением
\begin{align*}
    (\mathbb A y)(k) = \mathcal C_0 y(k + 1) + \mathcal C_1 y(k), \quad k \in \groupz, \; y \in l^p(\groupz; Y^N),
\end{align*}
где
\[
    \mathcal C_0 \sim \begin{pmatrix}
    I & 0 & \cdots &  0 \\
    0 & I  & \cdots &  0 \\
    \vdots & \vdots & \ddots &  \vdots \\
    0 & 0 & \cdots &  C_0
   \end{pmatrix},\quad
   \mathcal C_1 \sim \begin{pmatrix}
    0 & -I & 0  & \cdots & 0 & 0 \\
    0 & 0  & -I & \cdots & 0 & 0 \\
    0 & 0  & 0 & \cdots & 0 & 0 \\
    \vdots & \vdots & \vdots & \ddots & \vdots & \vdots \\
    0 & 0 & 0 & \cdots & 0 & -I \\
    C_N & C_{N-1} & C_{N-2} & \cdots & C_2 & C_1
   \end{pmatrix}.
\]

Введём в рассмотрение операторнозначную функцию $H \colon \groupt \to \End X$:
\[ H(\gamma) = \gamma^N C_0 + \gamma^{N-1} C_1 + \dotsc + C_N, \quad \gamma \in \groupt = \menge{\lambda \in \fieldc : \absv{\lambda} = 1}. \]
Эту функцию назовём \emph{характеристической функцией} оператора $ \mathcal A$. Множество $\rho(H)$, состоящее из таких $\gamma \in \groupt$, что оператор $H(\gamma)$ обратим, назовём \emph{резольвентным множеством} функции $H$, а дополнение к нему, $s(H) \hm= \groupt \setminus \rho(H)$ --- \emph{сингулярным множеством} этой функции.

\begin{theorem}\label{th:constant_inverse}
    Разностный оператор $\mathcal A$ с постоянными коэффициентами $C_i$, $i = \overline{0,N}$,обратим тогда и только тогда, когда сингулярное множество $s(H)$ его характеристической функции пусто. Если $s(H) = \varnothing$, то обратный оператор $\mathcal A^{-1} \in \End l^p$ представим в виде
    \begin{equation}\label{eq:invconst}
     (\mathcal A^{-1} x)(k) = (G * x)(k)= \sum_{n \in \groupz} G(k - n) x(n), \quad k \in \groupz, \; x \in l^p.
    \end{equation}
    Функция $G$ принадлежит банаховой алгебре $l^1(\groupz, \End Y)$ (со свёрткой функций в качестве умножения) и допускает представление вида
    \[ G(n) = \frac{1}{2 \pi} \int_{\groupt} (H(\gamma))^{-1} \gamma^n \dd \gamma, \quad n \in \groupz. \]
\end{theorem}

\begin{proof}
    Разностный оператор первого порядка $\mathbb A$ с постоянными коэффициентами обратим тогда и только тогда, когда сингулярное множество его характеристической функции $\mathcal H(\gamma) = \gamma \mathcal C_0 + \mathcal C_1$ пусто (иначе говоря, спектр линейного операторного пучка не содержит точек единичной окружности) \cite[теорема 3]{Bas01}.
    Запишем матрицу оператора $\mathcal H(\gamma)$:

    \[ \mathcal H(\gamma) \sim \begin{pmatrix}
    \gamma I & -I & 0  & \cdots & 0 & 0 \\
    0 & \gamma I  & -I & \cdots & 0 & 0 \\
    0 & 0  & \gamma I & \cdots & 0 & 0 \\
    \vdots & \vdots & \vdots & \ddots & \vdots & \vdots \\
    0 & 0 & 0 & \cdots & \gamma I & -I \\
    C_N & C_{N-1} & C_{N-2} & \cdots & C_2 & \gamma C_0 + C_1
   \end{pmatrix}.\]
   Из теоремы \ref{th:main_abstract} следует, что состояния обратимости операторов $\mathcal H(\gamma)$ и $H(\gamma)$ совпадают. Значит $s(H) \hm= s(\mathcal H)$. Отсюда получаем первое утверждение теоремы.

   Заметим, что $G(n)$ представляют собой коэффициенты Фурье функции $(H(\gamma))^{-1}$, которая является голоморфной в окрестности единичной окружности как резольвента полиномиального операторного пучка. Следовательно, её ряд Фурье сходится абсолютно, откуда и следует, что $G \hm\in l^1(\groupz, \End Y)$. Благодаря этому, оператор $\mathcal A^{-1}$, задаваемый формулой \eqref{eq:invconst}, определен корректно. Непосредственная проверка показывает, что этот оператор является обратным к оператору $\mathcal A$.
\end{proof}

\begin{comment}
 Покажем, что $ D^{-1}$ в самом деле является обратным.

   Правый обратный:
   \begin{align*}
    ( {D D}^{-1}x)(k) &= \sum_{i=0}^N C_i ( D^{-1} x)(k + N - i) = 
    \sum_{i=0}^N C_i (G * x)(k + N - i) = \sum_{i=0}^N C_i \left(\sum_{n \in \groupz} G(k+N-i-n) x(n)\right) = \\ &= \sum_{n \in \groupz} \left( \sum_{i = 0}^N C_i G(k + N - i - n) \right) x(n);
   \end{align*}
   \begin{align*}
    \sum_{i = 0}^N C_i G(k + N - i - n) &= \sum_{i=0}^N C_i \left( \int_{\groupt} (H(\gamma))^{-1} \gamma^{k + N - i - n} \dd \gamma \right) = \int_{\groupt} \sum_{i=0}^N C_i (H(\gamma))^{-1} \gamma^{k + N - i - n} \dd \gamma = \\ &= \int_{\groupt} \left(\sum_{i=0}^N \gamma^{N - i} C_i   \right) (H(\gamma))^{-1} \gamma^{k - n} \dd \gamma = \int_{\groupt} \gamma^{k-n} \dd \gamma = \delta_{nk};
   \end{align*}
   Значит,
   \[ ( {D D}^{-1}x)(k) = x(k) \]

   Левый обратный:
   \begin{align*}
    ({D}^{-1}  D x)(k) &= \sum_{n \in \groupz} G(k - n) ({D}x)(n) =
     \sum_{n \in \groupz} G(k - n) \left(\sum_{i=0}^N C_i x(n + N - i) \right) =
     \sum_{n \in \groupz} \sum_{i=0}^N  G(k - n) C_i x(n + N - i) = \\ &=
     \sum_{n \in \groupz} \sum_{i=0}^N G(k - n + N - i) C_i x(n) = \sum_{n \in \groupz}
     \left( \sum_{i=0}^N G(k + N - i - n) C_i \right) x(n) = x(k).
   \end{align*}
\end{comment}

\begin{assumption}\label{assum:bounded}
    Все решения однородного разностного уравнения
    \begin{equation}\label{eq:homogeneous}
        x(k + N) + C_1 x(k + N - 1) + \dotsc + C_N x(k) = 0,
    \end{equation}
    рассматриваемого на $\groupz_+$ ограничены.
\end{assumption}

В условиях предположения \ref{assum:bounded} любое решение $x \in l^\infty(\groupz_+, Y)$ однородного уравнения удовлетворяет равенствам
\[
\begin{pmatrix}
x(n) \\
x(n + 1) \\
\vdots \\
x(n + N - 2) \\
x(n + N - 1)
\end{pmatrix}
=
\begin{pmatrix}
    0 & I  & \cdots & 0 & 0 \\
    0 & 0  & \cdots & 0 & 0 \\
    \vdots & \vdots & \ddots & \vdots & \vdots \\
    0 & 0 & \cdots & 0 & I \\
    -C_N & -C_{N-1} & \cdots & -C_2 & -C_1
   \end{pmatrix}^{\displaystyle n}
   \begin{pmatrix}
    x(0) \\
    x(1) \\
    \vdots \\
    x(N - 2) \\
    x(N - 1)
   \end{pmatrix}.
   \]

Тогда из ограниченности всех решений однородного уравнения и теоремы Банаха\-Штейнгауза следует, что
\[ \sup_{n\geq 0} \norm{\mathcal C_1^n} = M(\mathcal C_1) < \infty.\]
Следовательно, спектральный радиус оператора $\mathcal C_1$ не превосходит единицы, т.е.
\[ \spectrum{\mathcal C_1} \subset \menge{\lambda \in \fieldc : \absv{\lambda} \geq 1}. \]

Теперь можно применить результат из \cite[теорема 1]{Bas15}:
\begin{theorem}\label{th:asymptotic}
    Пусть выполнены условия предположения \ref{assum:bounded} и 
    \[ \spectrum{\mathcal C_1} \cap \groupt = \menge{\gamma_1, \dotsc, \gamma_m}.\] 
    Тогда существуют операторнозначные функции $A_k \in l^\infty(\groupz_+, \End{Y^N})$, $k \hm= \overline{1,m}$, такие что для любого решения $x \colon \groupz_+ \to Y$ уравнения \ref{eq:homogeneous} имеют место следующие представления
    \[ (x(n), x(n+1), \dotsc, x(n+N-1)) = \left(\sum_{k=1}^m \gamma_k^n A_k(n) \right) (x(0), x(1), \dotsc, x(N-1)),\]
    \[n \in \groupz_+.\]
    Функции $A_k$, $k \hm= \overline{1,m}$ обладают следующими свойствами:
    \begin{enumerate}
    \item операторы $A_k(n) \in \End{Y^N}$, $n \in \groupz_+$ принадлежат наименьшей замкнутой подалгебре $\mathcal A_{\mathcal C_1}$ из $\End Y^N$, содержащей оператор $\mathcal C_1$;
    \item $\lim_{n\to \infty} {\norm{A_k(n+1) - A_k(n)}} = 0$;
    \item $\lim_{n\to \infty} {\norm{\mathcal C_1 A_k(n) - \gamma_k A_k(n)}} = 0$;
    \item $\lim_{n\to \infty} {\norm{A_k(n) A_j(n)}} = 0$ для $k \neq j$, $k, j = \overline{1,m}$.
    \end{enumerate}
\end{theorem}

В заключение отметим, что основные результаты работы (теоремы 1-5) имеют место
для разностных операторов, действующих в весовых пространствах
последовательностей векторов (см. статьи \cite{Bic10}, \cite{Bic13}, \cite{Bic14}).
