\intro
\section*{О содержании и структуре работы}
Из курса дифференциальных и разностных уравнений известен метод приведения уравнения $N$-ого порядка к системе из $N$ уравнений первого порядка. В данной работе рассматривается обобщение этого метода для исследования операторных полиномов с коэффициентами из банахова пространства. Исследование спектральных свойств операторных полиномов сводится к изучению спектральных свойств оператора, заданного операторной матрицей. Полученные результаты (теоремы \labelcref{th:main_abstract,th:inverse,th:abstract_fredholm}) применяются к разностным операторам высокого порядка. Получены условия их обратимости, фредгольмовости (теоремы \labelcref{th:stinvdiff,th:fredholm_difference,th:constant_inverse}), асимптотическое представление ограниченных решений однородного разностного уравнения (теорема \labelcref{th:asymptotic}).

Работа состоит из введения, трёх глав и заключения.

Во введении дается общая характеристика работы.

В первой главе приводятся основные определения и понятия, используемые в работе.

Во второй главе формулируются основные результаты работы и приводятся их доказательства.

В третьей главе основные результаты используются для исследования условий обратимости и фредгольмовости разностных операторов.

В заключении описаны возможные направления дальнейших исследований на данную тему.

\section*{История исследований}
Приводимая в работе конструкция перехода от изучения исходного операторного полинома $\mathcal A = C_0 A^N + C_1 A^{N - 1} + \dotsc + C_N \in \End X$ к изучению матричного оператора $\mathbb A \in \End X^N$, является непосредственным обобщением известного из курсов дифференциальных и разностных уравнений приёма сведения дифференциального или разностного уравнения $N$-ого порядка к системе из $N$ дифференциальных (разностных) уравнений. Для более специальных классов операторных полиномов аналог теоремы \ref{th:inverse} получен в монографиях А. Б. Антоневича \cite[теорема 9.1]{antonevich2}, \cite{antonevich}.

Теория разностных операторов первого порядка развивалась в~работах \cite{antonevich2,antonevich,kurbatov,kurbatov2,massera,henri,megan,dorogovtsev,chicone,inverse1992,memory2014,green2015,BasDup15,BasPas01,Bas00,Bas13,Bas15,Bic10,Bic13,Bic14}.

Понятие состояний обратимости используется, в частности, в статье \cite{Bas13}.

В отличие от статьи \cite{BasDup15}, где изучались разностные операторы второго порядка, оператор $C_0$ из представления операторного полинома $\mathcal A$ может быть необратимым оператором. Случай необратимого оператора при старшей степени оператора $A$ позволяет получать аналоги теорем \labelcref{th:main_abstract,th:inverse,th:abstract_fredholm} для случая операторного полинома $\mathcal A$, где оператор $A$ --- замкнутый оператор с непустым резольвентным множеством (в частности, дифференциальный оператор). Следует отметить, что такой прием не применим к дифференциальным операторам второго порядка, рассматриваемых в статьях \cite{shkalikov,HryShk03}. В данной работе предложен иной (более простой) способ доказательства основных результатов статьи \cite{BasDup15}. Он состоит в сопоставлении операторному полиному порядка $N$ оператора $\widetilde{\mathbb A}$, заданного операторной матрицей порядка $N+1$, который имеет то же множество состояний обратимости.
